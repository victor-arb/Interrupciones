
\documentclass[12pt, letter]{article}
\usepackage[utf8]{inputenc}
\usepackage[spanish,es-tabla]{babel}
%\usepackage{times},puede ser arial 
\usepackage{csquotes}
\usepackage[left=2.54cm, right=2.54cm,top=2.54cm,bottom=2.54cm]{geometry}
\renewcommand{\baselinestretch}{1.5}
\usepackage[backend=biber,style=apa]{biblatex}
\bibliography{Referencias.bib}
\usepackage{graphicx}
\usepackage{subcaption}
\usepackage[hidelinks]{hyperref}

\title{\huge{Interrupciones}}
\author{Victor Manuel Arbeláez Ramírez \\ Facultad de ingeniería \\ Universidad de Antioquia}
\date{}

\begin{document}\raggedright

\maketitle


\section*{Introducción}
Los microprocesadores son herramientas que pueden ejecutar una gran cantidad de instrucciones en un segundo, por lo que, para hacer uso del mayor potencial de sus capacidades, se opta eficiente por las ventajas que deja el sistema de las interrupciones como se podrá apreciar a continuación. Aun así, el concepto de éstas interrupciones es un poco más complejo de entender, debido a que es poco intuitivo a lo que se viene aprendiendo en los cursos básicos de computación y programación, en donde la mayoría de procesos se ejecuta de una manera continua. Se enunciará entonces, su definición, su historia, los tipos que existen y la forma de implementarlos teniendo como objetivo el conocer que son y para que se usan las interrupciones a nivel del microprocesador.
\newpage



\end{document}\raggedright
